% !TEX program = pdflatex
\documentclass[aspectratio=169,10pt]{beamer}

% ============================================================
%  Theme & Packages
% ============================================================
\usetheme{Madrid}
\usecolortheme{default}
\usepackage{booktabs}
\usepackage{amsmath,amssymb}
\usepackage{graphicx}
\usepackage{hyperref}
\usepackage{tikz}
\usepackage{multirow}

% Path to figures folder (one level up)
\graphicspath{{../figures/}}

% Custom footer
\setbeamertemplate{footline}{
  \leavevmode%
  \hbox{%
  \begin{beamercolorbox}[wd=.333333\paperwidth,ht=2.25ex,dp=1ex,center]{author in head/foot}%
    \usebeamerfont{author in head/foot}\insertshortinstitute
  \end{beamercolorbox}%
  \begin{beamercolorbox}[wd=.333333\paperwidth,ht=2.25ex,dp=1ex,center]{title in head/foot}%
    \usebeamerfont{title in head/foot}\insertshorttitle
  \end{beamercolorbox}%
  \begin{beamercolorbox}[wd=.333333\paperwidth,ht=2.25ex,dp=1ex,right]{date in head/foot}%
    \usebeamerfont{date in head/foot}\insertshortdate{}\hspace*{2em}
    \insertframenumber{} / \inserttotalframenumber\hspace*{2ex}
  \end{beamercolorbox}}%
  \vskip0pt%
}

\setbeamerfont{author}{size=\small}

% ============================================================
%  Title
% ============================================================
\title[Modelling VIX Dynamics]{Modelling VIX Dynamics: GARCH vs Compound Poisson}
\subtitle{Contrasting Volatility-Clustering and Jump-Driven Approaches}
\author{Vittorio Prana CHANDREAN (20896895) \newline CHONG Tin Tak (20920359) \newline CHOI Man Hou (20894196)}
\institute{HKUST - IEDA4000E}
\date{\today}

% ============================================================
\begin{document}

% ----------------------------------------------------------
\begin{frame}
\titlepage
\end{frame}

% ----------------------------------------------------------
\begin{frame}{Outline}
\tableofcontents
\end{frame}

% ============================================================
\section{Introduction}
% ============================================================

\begin{frame}{Objective}
\textbf{Compare two frameworks for VIX modeling:}
\begin{enumerate}
    \item Classical \textbf{GARCH-family models} on log-VIX (capturing volatility clustering)
    \item A \textbf{Compound Poisson Process (CPP)} for VIX shocks
\end{enumerate}

\vspace{1em}
\textbf{Scope:}
\begin{itemize}
    \item 15+ years of daily VIX (log) data
    \item Spanning calm and \textbf{crisis regimes}
    \item Fit each model and assess risk (VaR/CVaR)
\end{itemize}

\vspace{1em}
\textbf{Key Focus:}
\begin{itemize}
    \item Model fit and interpretability of parameters
    \item Risk implications (especially extreme moves)
\end{itemize}

\vspace{0.5em}
\textit{Volatility (GARCH) and jump (CPP) models are complementary: GARCH captures smooth variance dynamics, CPP targets tail events and separates shock timing/magnitude.}
\end{frame}

\begin{frame}{What is VIX?}
\begin{itemize}
    \item \textbf{VIX} = CBOE Volatility Index
    \item Measures the market's expectation of 30-day volatility (S\&P 500 options)
    \item Often called the ``\textbf{fear gauge}'' --- moves inversely to the S\&P 500
\end{itemize}

\vspace{1em}
\textbf{Applications:}
\begin{itemize}
    \item Risk management and hedging
    \item Derivatives pricing
    \item Portfolio allocation
    \item Market sentiment indicator
\end{itemize}
\end{frame}

\begin{frame}{Why Model Volatility?}
\textit{``Volatility drives option pricing \& risk, so understanding its behavior is crucial for hedging and forecasting''}

\vspace{1em}
\textbf{Empirical features:}
\begin{itemize}
    \item VIX exhibits \textbf{mean reversion} and heavy-tailed spikes during crises
    \item Modelling VIX dynamics helps in forecasting stress periods and hedging volatility exposure
\end{itemize}

\vspace{1em}
\textbf{Data note:}
\begin{itemize}
    \item We model \textbf{log-VIX} (or log-changes) to stabilize variance
    \item Analysis covers multiple regimes (pre-crisis, 2020 COVID crash, etc.)
\end{itemize}

\vspace{0.5em}
\textit{VIX reflects forward volatility; its large jumps in crises motivate heavy-tailed modelling.}
\end{frame}

% ============================================================
\section{Background \& Methodology}
% ============================================================

\begin{frame}{GARCH Models for Volatility}
\textbf{GARCH(1,1) specification:}
\[
a_t = \sigma_t \epsilon_t, \quad \sigma_t^2 = \omega + \alpha\, a_{t-1}^2 + \beta\, \sigma_{t-1}^2
\]

\vspace{0.5em}
\textbf{Behaviour:}
\begin{itemize}
    \item \textbf{Volatility clustering}: past large shocks $a_{t-1}^2$ increase current variance
    \item The \textbf{persistence} of volatility is governed by $\alpha + \beta$
    \item Values near 1 imply long memory
\end{itemize}

\vspace{0.5em}
\textbf{Squared-returns:}
\begin{itemize}
    \item $a_t^2$ follows an ARMA(1,1) process with AR coefficient $\phi = \alpha + \beta$
    \item Slow ACF decay when $\alpha + \beta \approx 1$
\end{itemize}

\vspace{0.5em}
\textbf{Fitting:} Use (quasi) MLE assuming Gaussian errors. Standardized residuals $\epsilon_t$ should be i.i.d.\ and uncorrelated.
\end{frame}

\begin{frame}{EGARCH: Modelling Asymmetry}
\textbf{EGARCH (Nelson, 1991):}
\[
\ln \sigma_t^2 = \omega + \beta \ln \sigma_{t-1}^2 + \alpha\bigl(|\epsilon_{t-1}| - \mathbb{E}|\epsilon|\bigr) + \gamma\, \epsilon_{t-1}
\]

\vspace{0.5em}
\textbf{Asymmetry:}
\begin{itemize}
    \item The term $\gamma \epsilon_{t-1}$ allows negative shocks ($\epsilon < 0$) to impact $\sigma_t^2$ differently
    \item Typically $\gamma < 0$ for financial data: bad news increases future vol more than good news
\end{itemize}

\vspace{0.5em}
\textbf{Advantages:}
\begin{itemize}
    \item Ensures $\sigma_t^2 > 0$ without parameter constraints
    \item Captures ``\textbf{leverage effect}'' --- higher sensitivity to negative returns
\end{itemize}

\vspace{0.5em}
\textbf{Note:} If $\gamma \approx 0$, EGARCH reduces to symmetric log-GARCH.
\end{frame}

\begin{frame}{Compound Poisson Process (CPP)}
\textbf{CPP formulation:} Let $N(t) \sim \text{Pois}(\lambda t)$ count shocks up to time $t$, and let $J_1, J_2, \ldots$ be i.i.d.\ positive shock sizes. The total shock impact is:
\[
\boxed{S(t) = \sum_{i=1}^{N(t)} J_i}, \quad S(0) = 0
\]

\vspace{0.5em}
\textbf{Components:}
\begin{itemize}
    \item $N(t)$: captures shock \textbf{arrivals} (\# days with extreme moves) at rate $\lambda$
    \item $J_i$: captures shock \textbf{magnitude} (absolute log-change in VIX)
\end{itemize}

\vspace{0.5em}
\textbf{Fitting:}
\begin{enumerate}
    \item Estimate $\lambda$ (annual jump rate) from counts
    \item Fit a distribution $F$ to observed jump sizes
    \item Simulate $S(1)$ (one year's total shocks) by sampling $N \sim \text{Pois}(\lambda)$ and drawing $J_i$ from $F$
\end{enumerate}
\end{frame}

% ============================================================
\section{Data Pipeline}
% ============================================================

\begin{frame}{Data Source \& Preparation}
\textbf{Data:}
\begin{itemize}
    \item Daily VIX index levels (CBOE) for 15+ years
    \item \textbf{Train:} 2010--2021; \textbf{Test:} 2022--2025
    \item Compute log-VIX and log-changes
\end{itemize}

\vspace{1em}
\textbf{Regimes:}
\begin{itemize}
    \item Pre-Crisis (2010--2019)
    \item COVID (2020 crash)
    \item Post-COVID (2021--2023 recovery)
    \item Recent (2024--2025)
\end{itemize}

\vspace{1em}
\textbf{Summary Stats:}
\begin{itemize}
    \item Expected mean-reversion in log-VIX
    \item Variance spikes in crises
    \item $\alpha + \beta \approx 1$ in GARCH indicates high persistence
\end{itemize}
\end{frame}

\begin{frame}{Log-VIX Time Series}
\begin{center}
    \includegraphics[width=0.85\textwidth]{vix_series.png}
\end{center}
\begin{itemize}
    \item Red markers indicate identified shock days (top 5\% of $\Delta\log$ VIX).
\end{itemize}
\end{frame}

% ============================================================
\section{Model Development \& Results}
% ============================================================

\begin{frame}{Volatility Model Comparison}
\begin{table}[ht]
\centering
\begin{tabular}{lcccc}
\toprule
Model & Distribution & AIC & Persistence & Half-life \\
\midrule
GARCH(1,1) & GED & 27,531 & 0.852 & 4.3 days \\
\textbf{EGARCH(1,1)} & GED & \textbf{27,395} & 0.934 & 10.2 days \\
\bottomrule
\end{tabular}
\end{table}

\vspace{0.5em}
\textbf{Key Findings:}
\begin{itemize}
    \item EGARCH achieves \textbf{lowest AIC} $\Rightarrow$ best in-sample fit
    \item Higher persistence in EGARCH $\Rightarrow$ shocks decay more slowly ($\approx$10 days half-life)
    \item GED distribution captures fat tails better than Normal or Student-t
\end{itemize}
\end{frame}

\begin{frame}{News Impact Curve (Asymmetry)}
\begin{center}
    \includegraphics[width=0.65\textwidth]{news_impact.png}
\end{center}
\begin{itemize}
    \item Positive shocks (VIX spikes) increase future variance more than negative shocks of equal magnitude.
    \item EGARCH captures this \textbf{leverage effect}.
\end{itemize}
\end{frame}

\begin{frame}{Q--Q Plot of Standardized Residuals}
\begin{center}
    \includegraphics[width=0.35\textwidth]{qq.png}
\end{center}
\begin{itemize}
    \item Points hug the 45° line in the tails $\Rightarrow$ GED captures fat tails well.
\end{itemize}
\end{frame}

% ============================================================
% COMPOUND POISSON PROCESS
% ============================================================

\begin{frame}{CPP: Jump Size Distribution Selection}
\begin{table}[ht]
\centering
\begin{tabular}{lcccc}
\toprule
Distribution & Parameters & AIC & KS Statistic & KS p-value \\
\midrule
Exponential & 1 & 412.3 & 0.142 & 0.003 \\
Gamma & 2 & 385.7 & 0.089 & 0.085 \\
Lognormal & 2 & 391.2 & 0.098 & 0.052 \\
\textbf{Pareto} & 2 & \textbf{378.4} & \textbf{0.061} & \textbf{0.42} \\
Weibull & 2 & 388.9 & 0.095 & 0.068 \\
\bottomrule
\end{tabular}
\end{table}

\vspace{0.5em}
\textbf{Selection:} Pareto provides the best fit (lowest AIC, highest KS p-value).
\end{frame}

\begin{frame}{CPP: Fitted Parameters}
\begin{table}[ht]
\centering
\begin{tabular}{lcc}
\toprule
Parameter & Value & Interpretation \\
\midrule
$\lambda$ & 12.64/year & Shock arrival rate \\
$\alpha$ (Pareto shape) & 2.50 & Tail index \\
$x_{\min}$ (Pareto scale) & 0.127 & Minimum shock size \\
$\mathbb{E}[J]$ & 0.211 & Mean jump size (21.1\% log-move) \\
$\text{Std}[J]$ & 0.189 & Jump size volatility \\
$\mathbb{E}[J^2]$ & 0.080 & Second moment (for variance) \\
\midrule
$\mathbb{E}[S(1)]$ & 2.67/year & Expected annual impact \\
$\text{Std}[S(1)]$ & 1.00/year & Annual impact volatility \\
VaR (95\%) & 4.24 & 95th percentile annual impact \\
CVaR (95\%) & 5.01 & Expected Shortfall \\
\bottomrule
\end{tabular}
\end{table}
\end{frame}

\begin{frame}{CPP: Jump Size Distribution}
\begin{center}
    \includegraphics[width=0.6\textwidth]{jump_distribution.png}
\end{center}
\begin{itemize}
    \item Pareto distribution captures heavy right tail of shock magnitudes.
    \item Most shocks are moderate; a few are extreme.
\end{itemize}
\end{frame}

\begin{frame}{CPP: Simulated Paths}
\begin{center}
    \includegraphics[width=0.7\textwidth]{cpp_paths.png}
\end{center}
\begin{itemize}
    \item Gray: 50 sample paths of cumulative annual shock impact.
    \item Shaded: 5--95\% and 25--75\% confidence bands.
    \item Red: Median trajectory.
\end{itemize}
\end{frame}

\begin{frame}{CPP: VaR and CVaR Distribution}
\begin{center}
    \includegraphics[width=0.6\textwidth]{cpp_var.png}
\end{center}
\begin{itemize}
    \item Distribution of annual cumulative shock impact from 10,000 Monte Carlo simulations.
    \item VaR (95\%) $= 4.24$: ``In 95\% of years, total impact $\le$ 4.24.''
    \item CVaR (95\%) $= 5.01$: ``Average impact in worst 5\% of years.''
\end{itemize}
\end{frame}

\begin{frame}{CPP: Regime Comparison}
\begin{table}[ht]
\centering
\small
\begin{tabular}{lccccc}
\toprule
Regime & $\lambda$/Year & $\mathbb{E}[J]$ & $\mathbb{E}[S]$/Year & VaR 95\% & CVaR 95\% \\
\midrule
Pre-Crisis & 12.3 & 0.209 & 2.57 & 4.15 & 4.92 \\
\textbf{COVID} & \textbf{17.3} & \textbf{0.262} & \textbf{4.53} & \textbf{7.44} & \textbf{9.65} \\
Post-COVID & 11.6 & 0.188 & 2.19 & 3.44 & 3.85 \\
Recent & 13.6 & 0.216 & 2.95 & 4.70 & 5.63 \\
\midrule
Full Sample & 12.6 & 0.211 & 2.67 & 4.24 & 5.01 \\
\bottomrule
\end{tabular}
\end{table}

\textbf{Key Result:} COVID regime exhibits:
\begin{itemize}
    \item \textbf{41\% higher arrival rate}: $\lambda_{\text{COVID}} = 17.3$ vs $\lambda_{\text{Pre}} = 12.3$
    \item \textbf{25\% larger mean jumps}: $\mathbb{E}[J]_{\text{COVID}} = 0.262$ vs $\mathbb{E}[J]_{\text{Pre}} = 0.209$
    \item \textbf{76\% higher expected annual impact}
    \item \textbf{Nearly double VaR}
\end{itemize}
\end{frame}

\begin{frame}{CPP: Regime Risk Comparison}
\begin{center}
    \includegraphics[width=0.7\textwidth]{cpp_regime.png}
\end{center}
\end{frame}

% ============================================================
% CPP OUT-OF-SAMPLE EVALUATION
% ============================================================

\begin{frame}{CPP: Out-of-Sample Evaluation}
\textbf{Train-Test Split:}
\begin{itemize}
    \item Training: January 2010 -- December 2021 (75\% of data, $\approx$3,100 observations)
    \item Test: January 2022 -- November 2025 (25\% of data, $\approx$1,036 observations)
\end{itemize}

\vspace{0.5em}
\textbf{Forecasting:}
\begin{align*}
\text{Shock Count Forecast:} \quad & \hat{N}(T) = \hat{\lambda} \cdot T \\
\text{Cumulative Impact Forecast:} \quad & \hat{S}(T) = \hat{\lambda} \cdot \hat{\mathbb{E}}[J] \cdot T \\
\text{Risk Bounds:} \quad & \text{VaR}_T = \text{VaR}_{1\,\text{year}} \times \frac{T}{252}
\end{align*}
\end{frame}

\begin{frame}{CPP: Out-of-Sample Results}
\begin{table}[ht]
\centering
\small
\begin{tabular}{lcc}
\toprule
Metric & Value & Notes \\
\midrule
\textit{Trained Parameters} & & \\
$\hat{\lambda}$ & 0.050/day & 12.6 shocks/year \\
$\hat{F}$ & Pareto & $\alpha = 2.50$, $x_{\min} = 0.127$ \\
$\hat{\mathbb{E}}[J]$ & 0.211 & Mean jump size \\
$\hat{\text{Std}}[J]$ & 0.189 & Jump volatility \\
\midrule
\textit{Test Period} & & \\
Test Days & 1,036 & Approx.\ 4 years \\
Actual Shocks & 63 & Observed \\
Predicted Shocks & 51.8 & $\hat{\lambda} \times 1036$ \\
Error & $-17.8\%$ & Underforecast \\
\midrule
Actual Impact & 13.4 & $\sum_i |J_i|$ \\
Predicted Impact & 10.9 & $\hat{\lambda} \cdot \hat{\mathbb{E}}[J] \cdot T$ \\
Error & $-18.5\%$ & Underforecast \\
\midrule
Scaled VaR 95\% & 15.2 & For test period \\
VaR Exceeded? & No & Actual $<$ VaR \\
\bottomrule
\end{tabular}
\end{table}
\end{frame}

\begin{frame}{CPP: Out-of-Sample Distribution}
\begin{center}
    \includegraphics[width=0.75\textwidth]{cpp_forecast.png}
\end{center}
\begin{itemize}
    \item Actual outcome at 72nd percentile of simulated distribution
    \item VaR bounds not exceeded $\Rightarrow$ risk measure is conservative
    \item Parameter stability: $\alpha \approx 2.6$ on test data alone confirms Pareto fit
\end{itemize}
\end{frame}

% ============================================================
\section{Discussion \& Conclusion}
% ============================================================

\begin{frame}{Key Findings}
\begin{enumerate}
    \item \textbf{GARCH Persistence:} VIX volatility shocks have a half-life of 4--10 days
    \item \textbf{EGARCH Best Fit:} Lowest AIC; captures asymmetric leverage effect
    \item \textbf{CPP Risk Quantification:} 
    \begin{itemize}
        \item Pareto-distributed jumps with $\alpha = 2.50$
        \item VaR 95\% $= 4.24$/year, CVaR 95\% $= 5.01$/year
    \end{itemize}
    \item \textbf{Regime Dependence:} COVID showed 76\% higher expected annual impact
    \item \textbf{CPP Out-of-Sample:} 
    \begin{itemize}
        \item $\sim$18\% forecast error (acceptable given unusual test period)
        \item VaR bounds respected; model well-calibrated
    \end{itemize}
\end{enumerate}
\end{frame}

\begin{frame}{Future Directions}
\textbf{Potential model extensions:}

\vspace{0.5em}
\textbf{Hawkes Processes:}
\begin{itemize}
    \item Allows shock arrivals to be \textbf{self-exciting} (clustering of jumps)
    \item Model aftershocks explicitly
\end{itemize}

\textbf{Hybrid Models:}
\begin{itemize}
    \item Combine GARCH with jump processes in one framework
\end{itemize}

\textbf{High-Frequency Data:}
\begin{itemize}
    \item Use intraday VIX futures or realized volatility to refine jump detection
\end{itemize}

\textbf{Multivariate VIX:}
\begin{itemize}
    \item Extend to joint modeling of VIX and other volatility indices for co-movements and contagion
\end{itemize}

\textbf{Machine Learning:}
\begin{itemize}
    \item Employ regime-switching ML or nonparametric methods to detect shifts in $\lambda$ and $\alpha$
\end{itemize}
\end{frame}

\begin{frame}{Conclusion}
\begin{itemize}
    \item VIX exhibits \textbf{persistent, asymmetric volatility clustering} well captured by EGARCH
    \item \textbf{Compound Poisson Process} quantifies aggregate shock risk:
    \begin{itemize}
        \item VaR 95\% $= 4.24$/year
        \item CVaR 95\% $= 5.01$/year
    \end{itemize}
    \item \textbf{COVID regime} showed 76\% higher expected annual impact than baseline
    \item \textbf{CPP out-of-sample}: $\sim$18\% forecast error; VaR bounds respected; well-calibrated
    \item GARCH and CPP are \textbf{complementary}: GARCH for smooth variance dynamics, CPP for tail events
\end{itemize}

\vspace{1em}
\centering
\Large \textbf{Thank you!}\\[0.5em]
\normalsize Questions?
\end{frame}

% ============================================================
\section*{Appendix}
% ============================================================

\begin{frame}{Appendix: Key Equations}
\textbf{GARCH(1,1):}
\[
\sigma_t^2 = \omega + \alpha\, a_{t-1}^2 + \beta\, \sigma_{t-1}^2
\]

\textbf{EGARCH(1,1):}
\[
\ln \sigma_t^2 = \omega + \beta \ln \sigma_{t-1}^2 + \alpha\bigl(|\epsilon_{t-1}| - \mathbb{E}|\epsilon|\bigr) + \gamma\, \epsilon_{t-1}
\]

\textbf{Compound Poisson Process:}
\[
S(T) = \sum_{i=1}^{N(T)} J_i, \quad N(T) \sim \mathrm{Poisson}(\lambda T), \quad J_i \sim F
\]

\textbf{Expected Annual Impact:}
\[
\mathbb{E}[S] = \lambda \cdot \mathbb{E}[J] \cdot T
\]
\end{frame}

\begin{frame}{Appendix: References}
\small
\begin{itemize}
    \item Bollerslev, T. (1986). Generalized autoregressive conditional heteroskedasticity. \textit{J.\ Econometrics}.
    \item Nelson, D. B. (1991). Conditional heteroskedasticity in asset returns. \textit{Econometrica}.
    \item Cont, R., \& Tankov, P. (2004). \textit{Financial Modelling with Jump Processes}. Chapman \& Hall/CRC.
    \item McNeil, A. J., Frey, R., \& Embrechts, P. (2015). \textit{Quantitative Risk Management}. Princeton University Press.
\end{itemize}
\end{frame}

% ============================================================
\end{document}
