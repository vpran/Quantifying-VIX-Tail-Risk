% !TEX program = pdflatex
\documentclass[aspectratio=169,10pt]{beamer}

% ============================================================
%  Theme & Packages
% ============================================================
\usetheme{Madrid}
\usecolortheme{default}
\usepackage{booktabs}
\usepackage{amsmath,amssymb}
\usepackage{graphicx}
\usepackage{hyperref}
\usepackage{tikz}
\usepackage{multirow}

% Path to figures folder (one level up)
\graphicspath{{../figures/}}

% Custom footer: show only institute (not author)
\setbeamertemplate{footline}{
  \leavevmode%
  \hbox{%
  \begin{beamercolorbox}[wd=.333333\paperwidth,ht=2.25ex,dp=1ex,center]{author in head/foot}%
    \usebeamerfont{author in head/foot}\insertshortinstitute
  \end{beamercolorbox}%
  \begin{beamercolorbox}[wd=.333333\paperwidth,ht=2.25ex,dp=1ex,center]{title in head/foot}%
    \usebeamerfont{title in head/foot}\insertshorttitle
  \end{beamercolorbox}%
  \begin{beamercolorbox}[wd=.333333\paperwidth,ht=2.25ex,dp=1ex,right]{date in head/foot}%
    \usebeamerfont{date in head/foot}\insertshortdate{}\hspace*{2em}
    \insertframenumber{} / \inserttotalframenumber\hspace*{2ex}
  \end{beamercolorbox}}%
  \vskip0pt%
}

% Smaller author font on title page
\setbeamerfont{author}{size=\small}

% ============================================================
%  Title
% ============================================================
\title[VIX Shock Persistence \& Frequency]{Shock Persistence and Shock Frequency in VIX}
\subtitle{A Quantitative Analysis of Volatility Dynamics}
\author{CHONG Tin Tak (20920359) \newline CHOI Man Hou (20894196) \newline Vittorio Prana CHANDREAN (20896895)}
\institute{HKUST - IEDA4000E}
\date{\today}

% ============================================================
\begin{document}

% ----------------------------------------------------------
\begin{frame}
\titlepage
\end{frame}

% ----------------------------------------------------------
\begin{frame}{Outline}
\tableofcontents
\end{frame}

% ============================================================
\section{Introduction}
% ============================================================

\begin{frame}{What is VIX?}
\begin{itemize}
    \item \textbf{VIX} = CBOE Volatility Index, derived from S\&P~500 option prices.
    \item Often called the ``fear gauge'' --- rises when markets expect turbulence.
    \item Understanding VIX dynamics is crucial for:
    \begin{itemize}
        \item Risk management and hedging
        \item Derivatives pricing
        \item Portfolio allocation
    \end{itemize}
\end{itemize}
\end{frame}

\begin{frame}{Research Questions}
\begin{enumerate}
    \item \textbf{How persistent is volatility?}\\
          How long does a VIX shock take to decay?
    \item \textbf{How frequently do large spikes occur?}\\
          Can we model extreme events as a point process?
    \item \textbf{Do shocks cluster?}\\
          Is there self-excitation in shock arrivals?
    \item \textbf{How do regimes affect shock dynamics?}\\
          Do crisis periods show different behavior?
    \item \textbf{Can we forecast VIX volatility?}\\
          Do GARCH-type models beat simple baselines out-of-sample?
\end{enumerate}
\end{frame}

% ============================================================
\section{Data}
% ============================================================

\begin{frame}{Data Overview}
\begin{itemize}
    \item \textbf{Source:} Yahoo Finance (ticker \texttt{\^{}VIX})
    \item \textbf{Period:} January 2010 -- November 2025
    \item \textbf{Observations:} 4,148 business days
    \item \textbf{Pre-processing:}
    \begin{itemize}
        \item Forward-fill missing dates
        \item 0.1\% winsorization to limit outlier influence
        \item Compute $\log(\text{VIX})$ and daily log-changes $\Delta\log(\text{VIX})$
    \end{itemize}
\end{itemize}
\end{frame}

\begin{frame}{VIX Time Series}
\begin{center}
    \includegraphics[width=0.85\textwidth]{vix_series.png}
\end{center}
\begin{itemize}
    \item Red markers indicate identified shock days (top 5\% of $\Delta\log$ VIX).
\end{itemize}
\end{frame}

% ============================================================
\section{Methodology}
% ============================================================

\begin{frame}{Volatility Models: GARCH Family}
\textbf{GARCH(1,1):}
\[
\sigma_t^2 = \omega + \alpha\,\varepsilon_{t-1}^2 + \beta\,\sigma_{t-1}^2
\]

\textbf{EGARCH(1,1):}
\[
\ln\sigma_t^2 = \omega + \beta\ln\sigma_{t-1}^2 + \alpha\bigl(|z_{t-1}| - \mathbb{E}|z|\bigr) + \gamma\, z_{t-1}
\]

\textbf{GJR-GARCH(1,1):}
\[
\sigma_t^2 = \omega + (\alpha + \gamma\, \mathbf{1}_{\varepsilon_{t-1}<0})\,\varepsilon_{t-1}^2 + \beta\,\sigma_{t-1}^2
\]
\begin{itemize}
    \item GARCH: Symmetric; EGARCH/GJR: Asymmetric (leverage effect via $\gamma$).
    \item Distribution: Auto-selected (Normal, Student-t, GED) via PIT diagnostics.
\end{itemize}
\end{frame}

\begin{frame}{HAR-RV: Realized Volatility Model}
\textbf{Heterogeneous Autoregressive Realized Volatility (Corsi, 2009):}
\[
RV_{t+1} = \beta_0 + \beta_d\, RV_t + \beta_w\, \overline{RV}_{t}^{(w)} + \beta_m\, \overline{RV}_{t}^{(m)} + \varepsilon_{t+1}
\]
where:
\begin{itemize}
    \item $RV_t = r_t^2$ (proxy for realized variance)
    \item $\overline{RV}_{t}^{(w)} = \frac{1}{5}\sum_{i=0}^{4} RV_{t-i}$ (weekly average)
    \item $\overline{RV}_{t}^{(m)} = \frac{1}{22}\sum_{i=0}^{21} RV_{t-i}$ (monthly average)
\end{itemize}
\vspace{0.5em}
\textbf{Use:} Baseline model capturing heterogeneous investor horizons.
\end{frame}

\begin{frame}{Shock Identification: Two Approaches}
\textbf{Method 1: Quantile-Based (Fixed Threshold)}
\[
\text{Shock}_t = \mathbf{1}\bigl\{\Delta\log(\text{VIX})_t \ge Q_{0.95}\bigr\}
\]
\begin{itemize}
    \item Simple, interpretable: top 5\% of daily changes.
\end{itemize}

\vspace{1em}
\textbf{Method 2: Volatility-Relative (Surprise-Based)}
\[
\text{Shock}_t = \mathbf{1}\bigl\{|r_t| > k \cdot \sigma_t\bigr\}, \quad k=2
\]
\begin{itemize}
    \item Captures ``surprises'' relative to expected volatility.
    \item More meaningful in high-volatility regimes.
\end{itemize}
\end{frame}

\begin{frame}{Point Process Models for Shock Arrivals}
\textbf{Homogeneous Poisson Process (HPP):}
\[
\lambda(t) = \lambda \quad \text{(constant rate)}
\]

\textbf{Non-Homogeneous Poisson Process (NHPP):}
\[
\lambda(t) = \exp\bigl(\beta_0 + \beta_1 \cdot \text{lagged\_log\_VIX}_t\bigr)
\]

\textbf{Hawkes Self-Exciting Process:}
\[
\lambda(t) = \mu + \sum_{t_i < t} \alpha \cdot e^{-\beta(t - t_i)}
\]
\begin{itemize}
    \item Each shock temporarily increases intensity $\Rightarrow$ \textbf{clustering}.
    \item \textbf{Branching ratio} $= \alpha/\beta < 1 \Rightarrow$ stationary.
\end{itemize}
\end{frame}

\begin{frame}{Forecast Evaluation Framework}
\begin{itemize}
    \item \textbf{Out-of-Sample Design:}
    \begin{itemize}
        \item Train on first 75\% of data (through Dec 2021).
        \item Monthly rolling re-estimation of GARCH.
        \item Forecast 1-step-ahead variance into the remaining 25\%.
    \end{itemize}
    \item \textbf{Baselines:}
    \begin{itemize}
        \item EWMA ($\lambda = 0.94$)
        \item 63-day rolling variance
        \item HAR-RV model
    \end{itemize}
    \item \textbf{Metrics:}
    \begin{itemize}
        \item Log-score (predictive density evaluation)
        \item 95\% coverage rate
        \item PIT histogram (calibration diagnostic)
        \item Diebold--Mariano test (statistical significance)
    \end{itemize}
\end{itemize}
\end{frame}

% ============================================================
\section{Results}
% ============================================================

\begin{frame}{Volatility Model Comparison}
\begin{table}[ht]
\centering
\begin{tabular}{lccccc}
\toprule
Model & Distribution & AIC & Persistence & Half-life & Leverage ($\gamma$) \\
\midrule
GARCH(1,1) & GED & 27,531 & 0.852 & 4.3 days & -- \\
\textbf{EGARCH(1,1)} & GED & \textbf{27,395} & 0.934 & 10.2 days & Yes \\
GJR-GARCH(1,1) & GED & 27,447 & 0.866 & 4.8 days & $-0.27$ \\
\bottomrule
\end{tabular}
\end{table}
\begin{itemize}
    \item EGARCH achieves lowest AIC $\Rightarrow$ best in-sample fit.
    \item Higher persistence in EGARCH $\Rightarrow$ shocks decay more slowly ($\approx$10 days half-life).
    \item GJR-GARCH $\gamma < 0$: negative returns increase volatility more.
\end{itemize}
\end{frame}

\begin{frame}{HAR-RV Results}
\begin{table}[ht]
\centering
\begin{tabular}{lcc}
\toprule
Component & Coefficient & Interpretation \\
\midrule
$\beta_{\text{daily}}$ & 0.117 & Short-term (1-day) effect \\
$\beta_{\text{weekly}}$ & 0.231 & Medium-term (5-day) effect \\
$\beta_{\text{monthly}}$ & 0.042 & Long-term (22-day) effect \\
\midrule
$R^2$ & 0.049 & Explanatory power \\
\bottomrule
\end{tabular}
\end{table}
\begin{itemize}
    \item Weekly component dominates $\Rightarrow$ swing traders' horizon matters most.
    \item Low $R^2$ expected: squared returns are noisy proxies for true variance.
    \item HAR-RV provides parsimonious baseline for comparison.
\end{itemize}
\end{frame}

\begin{frame}{News Impact Curve (Asymmetry)}
\begin{center}
    \includegraphics[width=0.7\textwidth]{news_impact.png}
\end{center}
\begin{itemize}
    \item Positive shocks (VIX spikes) increase future variance more than negative shocks of equal magnitude.
\end{itemize}
\end{frame}

\begin{frame}{Q--Q Plot of Standardized Residuals}
\begin{center}
    \includegraphics[width=0.35\textwidth]{qq.png}
\end{center}
\begin{itemize}
    \item Points hug the 45° line in the tails $\Rightarrow$ GED captures fat tails well.
\end{itemize}
\end{frame}

\begin{frame}{Shock Statistics: Two Methods Compared}
\begin{table}[ht]
\centering
\begin{tabular}{lcc}
\toprule
Metric & Quantile (95\%) & Vol-Relative ($2\sigma$) \\
\midrule
Threshold (avg) & 0.127 & 0.147 \\
Total shocks & 208 & 207 \\
Rate (shocks/year) & 9.0 & 9.0 \\
\bottomrule
\end{tabular}
\end{table}
\begin{itemize}
    \item Both methods identify $\sim$9 shocks/year.
    \item Vol-relative threshold adapts to market conditions.
    \item NHPP: Lagged $\log$ VIX coefficient $= -0.16$ $\Rightarrow$ low VIX predicts fewer shocks.
\end{itemize}
\end{frame}

\begin{frame}{Hawkes Self-Exciting Process}
\begin{table}[ht]
\centering
\begin{tabular}{lcc}
\toprule
Parameter & Value & Interpretation \\
\midrule
Baseline $\mu$ & 0.027/day & Background intensity \\
Excitation $\alpha$ & 0.043 & Jump after each shock \\
Decay $\beta$ & 0.183 & Rate of decay \\
Branching ratio & 0.23 & Fraction triggered by past shocks \\
Half-life & 3.8 days & Excitation halving time \\
\bottomrule
\end{tabular}
\end{table}
\begin{itemize}
    \item Branching ratio $< 1$ $\Rightarrow$ process is \textbf{stationary}.
    \item $\approx$23\% of shocks are ``triggered'' by previous shocks.
    \item Excitation decays with $\sim$4-day half-life.
\end{itemize}
\end{frame}

\begin{frame}{Hawkes Intensity Over Time}
\begin{center}
    \includegraphics[width=0.85\textwidth]{hawkes_intensity.png}
\end{center}
\begin{itemize}
    \item Intensity spikes after each shock, then decays exponentially.
    \item Clustering visible during crisis periods.
\end{itemize}
\end{frame}

% ============================================================
% COMPOUND POISSON PROCESS
% ============================================================

\begin{frame}{Compound Poisson Process: Motivation}
\textbf{Problem:} HPP/NHPP/Hawkes model \emph{when} shocks occur, but not \emph{how big}.

\vspace{0.5em}
\textbf{Compound Poisson Process (CPP):}
\[
S(T) = \sum_{i=1}^{N(T)} J_i
\]
where:
\begin{itemize}
    \item $N(T) \sim \mathrm{Poisson}(\lambda T)$: number of shocks by time $T$
    \item $J_i \sim F$: random jump sizes (iid)
    \item $S(T)$: cumulative shock impact over horizon $T$
\end{itemize}

\vspace{0.5em}
\textbf{Key Insight:} Models \emph{both} timing and magnitude $\Rightarrow$ risk quantification.
\end{frame}

\begin{frame}{CPP: Jump Size Distribution Selection}
\textbf{Candidates for jump size distribution $F$:}
\begin{itemize}
    \item Exponential: $f(x) = \lambda e^{-\lambda x}$ (memoryless)
    \item Gamma: $f(x) \propto x^{k-1} e^{-x/\theta}$ (flexible shape)
    \item Lognormal: $\ln(J) \sim N(\mu, \sigma^2)$ (multiplicative)
    \item \textbf{Pareto}: $f(x) = \frac{\alpha x_m^\alpha}{x^{\alpha+1}}$ (heavy tails)
    \item Weibull: $f(x) \propto x^{k-1} e^{-(x/\lambda)^k}$ (hazard rate)
\end{itemize}

\vspace{0.5em}
\textbf{Selection via AIC + KS test:}
\begin{itemize}
    \item Best fit: \textbf{Pareto} ($\alpha = 2.50$, $x_{\min} = 0.127$)
    \item KS p-value $= 0.42$ $\Rightarrow$ cannot reject fit
\end{itemize}
\end{frame}

\begin{frame}{CPP: Fitted Parameters}
\begin{table}[ht]
\centering
\begin{tabular}{lcc}
\toprule
Parameter & Value & Interpretation \\
\midrule
$\lambda$ & 12.64/year & Shock arrival rate \\
$\mathbb{E}[J]$ & 0.211 & Mean jump size (21\% log-move) \\
$\mathrm{Std}[J]$ & 0.189 & Jump size volatility \\
$\mathbb{E}[S]$ & 2.67/year & Expected annual impact \\
\midrule
VaR (95\%) & 4.24 & 95th percentile annual impact \\
CVaR (95\%) & 5.01 & Expected Shortfall (tail avg) \\
\bottomrule
\end{tabular}
\end{table}
\begin{itemize}
    \item $\mathbb{E}[S] = \lambda \cdot \mathbb{E}[J] = 12.64 \times 0.211 = 2.67$
    \item VaR/CVaR computed via 10,000 Monte Carlo simulations.
\end{itemize}
\end{frame}

\begin{frame}{CPP: Jump Size Distribution}
\begin{center}
    \includegraphics[width=0.6\textwidth]{jump_distribution.png}
\end{center}
\begin{itemize}
    \item Pareto distribution captures heavy right tail of shock magnitudes.
    \item Most shocks are moderate; a few are extreme.
\end{itemize}
\end{frame}

\begin{frame}{CPP: Simulated Paths}
\begin{center}
    \includegraphics[width=0.7\textwidth]{cpp_paths.png}
\end{center}
\begin{itemize}
    \item Gray: 50 sample paths of cumulative annual shock impact.
    \item Shaded: 5--95\% and 25--75\% confidence bands.
    \item Red: Median trajectory.
\end{itemize}
\end{frame}

\begin{frame}{CPP: VaR and CVaR Distribution}
\begin{center}
    \includegraphics[width=0.6\textwidth]{cpp_var.png}
\end{center}
\begin{itemize}
    \item Distribution of annual cumulative shock impact.
    \item VaR (95\%) $= 4.24$: ``In 95\% of years, total impact $\le$ 4.24.''
    \item CVaR (95\%) $= 5.01$: ``Average impact in worst 5\% of years.''
\end{itemize}
\end{frame}

\begin{frame}{CPP: Regime Comparison}
\begin{table}[ht]
\centering
\small
\begin{tabular}{lcccccc}
\toprule
Regime & $\lambda$/Year & $\mathbb{E}[J]$ & $\mathbb{E}[S]$/Year & VaR 95\% & CVaR 95\% \\
\midrule
Pre-Crisis & 12.3 & 0.209 & 2.57 & 4.15 & 4.92 \\
\textbf{COVID} & \textbf{17.3} & \textbf{0.262} & \textbf{4.53} & \textbf{7.44} & \textbf{9.65} \\
Post-COVID & 11.6 & 0.188 & 2.19 & 3.44 & 3.85 \\
Recent & 13.6 & 0.216 & 2.95 & 4.70 & 5.63 \\
\bottomrule
\end{tabular}
\end{table}
\begin{itemize}
    \item COVID: Both higher arrival rate AND larger jumps $\Rightarrow$ 76\% higher $\mathbb{E}[S]$.
    \item VaR nearly doubles during crisis periods.
\end{itemize}
\end{frame}

\begin{frame}{CPP: Regime Risk Comparison}
\begin{center}
    \includegraphics[width=0.7\textwidth]{cpp_regime.png}
\end{center}
\end{frame}

% ============================================================
% CPP OUT-OF-SAMPLE EVALUATION
% ============================================================

\begin{frame}{CPP: Out-of-Sample Evaluation Framework}
\textbf{Objective:} Test if CPP trained on historical data can forecast future shock dynamics.

\vspace{0.5em}
\textbf{Train-Test Split:}
\begin{itemize}
    \item Training: First 75\% of data (Jan 2010 -- Dec 2021)
    \item Test: Remaining 25\% (Jan 2022 -- Nov 2025)
\end{itemize}

\vspace{0.5em}
\textbf{Forecasting Approach:}
\begin{enumerate}
    \item Fit CPP on training data: estimate $\lambda$, $F$ (jump distribution)
    \item For test period of $T$ days:
    \begin{itemize}
        \item Predicted shocks: $\hat{N} = \lambda \cdot T$
        \item Predicted impact: $\hat{S} = \lambda \cdot \mathbb{E}[J] \cdot T$
    \end{itemize}
    \item Compare predictions to actual test period outcomes
\end{enumerate}

\vspace{0.5em}
\textbf{Evaluation Metrics:}
\begin{itemize}
    \item Shock count error: $\frac{\hat{N} - N_{\text{actual}}}{N_{\text{actual}}}$
    \item Impact error: $\frac{\hat{S} - S_{\text{actual}}}{S_{\text{actual}}}$
    \item VaR exceedance: Did actual impact exceed predicted VaR 95\%?
\end{itemize}
\end{frame}

\begin{frame}{CPP: Out-of-Sample Results}
\begin{table}[ht]
\centering
\begin{tabular}{lcc}
\toprule
Metric & Value & Interpretation \\
\midrule
Training Period & 2010--2021 & 75\% of data \\
Test Period & 2022--2025 & 25\% of data \\
\midrule
$\lambda$ (trained) & 0.050/day & 12.6 shocks/year \\
Jump Dist (trained) & Pareto & $\alpha = 2.50$ \\
$\mathbb{E}[J]$ (trained) & 0.211 & Mean jump size \\
\midrule
Test Shocks (Actual) & 63 & 1,036 test days \\
Test Shocks (Predicted) & 51.8 & $\lambda \times 1036$ \\
Shock Count Error & $-17.8\%$ & Slight underforecast \\
\midrule
Test Impact (Actual) & 13.4 & Cumulative $|r|$ \\
Test Impact (Predicted) & 10.9 & $\lambda \cdot \mathbb{E}[J] \cdot T$ \\
Impact Error & $-18.5\%$ & Slight underforecast \\
\bottomrule
\end{tabular}
\end{table}
\begin{itemize}
    \item Model slightly underpredicts $\Rightarrow$ test period had above-average shock activity (2022 rate hikes, 2024 volatility).
\end{itemize}
\end{frame}

\begin{frame}{CPP: Out-of-Sample Distribution}
\begin{center}
    \includegraphics[width=0.75\textwidth]{cpp_forecast.png}
\end{center}
\begin{itemize}
    \item Gray: Simulated test period impacts using trained CPP.
    \item Red dashed: Actual test period impact (13.4).
    \item Actual falls within the bulk of the distribution $\Rightarrow$ model well-calibrated.
\end{itemize}
\end{frame}

\begin{frame}{CPP: Key Out-of-Sample Findings}
\begin{enumerate}
    \item \textbf{Reasonable Forecasts:} $\sim$18\% underforecast error is acceptable given test period included 2022 Fed rate hikes and 2024 August volatility spike.
    
    \item \textbf{Distribution Calibration:} Actual outcome at 72nd percentile of simulated distribution---model is not overconfident.
    
    \item \textbf{VaR Coverage:} Actual impact (13.4) did NOT exceed scaled VaR 95\% (15.2 for test period) $\Rightarrow$ risk measure is conservative.
    
    \item \textbf{Parameter Stability:} Pareto distribution with $\alpha \approx 2.5$ remains appropriate for test period shocks.
\end{enumerate}

\vspace{0.5em}
\textbf{Implication:} CPP provides reliable out-of-sample forecasts for shock risk quantification. Model can be used for forward-looking risk management with appropriate uncertainty bounds.
\end{frame}

\begin{frame}{Shock Magnitudes Over Time}
\begin{center}
    \includegraphics[width=0.6\textwidth]{shock_magnitudes.png}
\end{center}
\begin{itemize}
    \item Top: Individual shock magnitudes with 20-shock rolling mean.
    \item Bottom: Cumulative shock count shows acceleration during crises.
\end{itemize}
\end{frame}

\begin{frame}{Regime Analysis: Shock Rates Across Periods}
\begin{table}[ht]
\centering
\begin{tabular}{lccccc}
\toprule
Regime & Period & Obs & Shocks & Rate/Year & Ann.\ Vol \\
\midrule
Pre-Crisis & 2010--2019 & 2,606 & 127 & 12.3 & 1.21 \\
\textbf{COVID} & 2020 & 262 & 18 & \textbf{17.3} & \textbf{1.36} \\
Post-COVID & 2021--2023 & 781 & 36 & 11.6 & 1.12 \\
Recent & 2024--2025 & 499 & 27 & 13.6 & 1.34 \\
\midrule
Full Sample & 2010--2025 & 4,148 & 208 & 12.6 & 1.22 \\
\bottomrule
\end{tabular}
\end{table}
\begin{itemize}
    \item COVID period: \textbf{41\% higher} shock rate (17.3 vs 12.3/year).
    \item Volatility elevated across all crisis periods.
\end{itemize}
\end{frame}

\begin{frame}{Regime Comparison Visualization}
\begin{center}
    \includegraphics[width=0.85\textwidth]{regime_comparison.png}
\end{center}
\begin{itemize}
    \item Bar chart shows clear elevation during COVID and recent volatility.
\end{itemize}
\end{frame}

\begin{frame}{Monthly Shock Counts}
\begin{center}
    \includegraphics[width=0.85\textwidth]{shock_counts.png}
\end{center}
\begin{itemize}
    \item Notable clustering: 2011--12 (Euro crisis), 2018 (Volmageddon), 2020 (COVID), 2022 (rate hikes).
\end{itemize}
\end{frame}

\begin{frame}{Inter-Arrival Time Distribution}
\begin{center}
    \includegraphics[width=0.7\textwidth]{interarrival.png}
\end{center}
\begin{itemize}
    \item Histogram vs.\ exponential PDF: reasonable fit validates Poisson assumption.
\end{itemize}
\end{frame}

\begin{frame}{Forecast Evaluation: Log-Scores}
\begin{table}[ht]
\centering
\begin{tabular}{lc}
\toprule
Model & Log-Score (higher = better) \\
\midrule
GARCH & 1.275 \\
\textbf{EWMA} & \textbf{1.376} \\
Rolling Var & 1.276 \\
HAR-RV & 1.271 \\
\bottomrule
\end{tabular}
\end{table}
\begin{itemize}
    \item EWMA slightly outperforms GARCH out-of-sample.
    \item Diebold--Mariano $p < 0.001$ $\Rightarrow$ difference is statistically significant.
    \item HAR-RV comparable to rolling variance baseline.
\end{itemize}
\end{frame}

\begin{frame}{Cumulative Log-Score Difference}
\begin{center}
    \includegraphics[width=0.8\textwidth]{cum_loss_diff.png}
\end{center}
\begin{itemize}
    \item Downward trend: EWMA consistently accumulates better scores.
    \item GARCH rarely catches up, even briefly.
\end{itemize}
\end{frame}

\begin{frame}{PIT Histogram (Calibration)}
\begin{center}
    \includegraphics[width=0.45\textwidth]{pit.png}
\end{center}
\begin{itemize}
    \item Near-uniform distribution indicates well-calibrated density forecasts.
    \item Mean $\approx 0.51$, std $\approx 0.26$ (ideal: 0.5, 0.29).
    \item 95\% coverage: 95.0\% (at nominal).
\end{itemize}
\end{frame}

% ============================================================
\section{Discussion}
% ============================================================

\begin{frame}{Key Findings}
\begin{enumerate}
    \item \textbf{Persistence:} VIX volatility shocks have a half-life of 4--10 days; EGARCH captures longer memory.
    \item \textbf{Leverage Effect:} GJR-GARCH $\gamma = -0.27$ confirms asymmetric response.
    \item \textbf{Self-Excitation:} Hawkes branching ratio $= 0.23$ $\Rightarrow$ shocks trigger more shocks.
    \item \textbf{Compound Poisson:} Pareto-distributed jumps; VaR 95\% $= 4.24$/year.
    \item \textbf{Regime Dependence:} COVID showed 76\% higher expected annual impact.
    \item \textbf{CPP Out-of-Sample:} $\sim$18\% forecast error; VaR bounds respected; well-calibrated.
    \item \textbf{Forecasting:} EWMA beats GARCH on log-score; both well-calibrated.
\end{enumerate}
\end{frame}

\begin{frame}{Practical Implications}
\begin{itemize}
    \item \textbf{Risk Management:}
    \begin{itemize}
        \item Use EGARCH/GJR-GARCH for asymmetric VaR calculations.
        \item Hawkes intensity provides real-time ``shock alert'' signals.
        \item CPP gives VaR/CVaR for aggregate annual shock impact.
    \end{itemize}
    \item \textbf{Option Pricing:}
    \begin{itemize}
        \item Leverage effect matters: adjust for skewed vol-of-vol.
        \item Pareto jumps justify heavy-tail adjustments in pricing.
    \end{itemize}
    \item \textbf{Portfolio Allocation:}
    \begin{itemize}
        \item Regime detection enables adaptive hedging strategies.
        \item CPP regime analysis shows crisis periods need 2$\times$ risk budget.
    \end{itemize}
\end{itemize}
\end{frame}

\begin{frame}{Limitations}
\begin{itemize}
    \item GARCH re-estimation is computationally expensive; rolling used here.
    \item EWMA's superiority may reflect VIX's strong mean-reversion.
    \item NHPP covariates limited to lagged VIX; macro indicators could improve.
    \item Hawkes estimation assumes exponential decay; other kernels possible.
    \item Daily data only; intraday dynamics not captured.
\end{itemize}
\end{frame}

\begin{frame}{Future Work}
\begin{itemize}
    \item Incorporate regime-switching GARCH (Markov-switching) models.
    \item Test realized volatility from high-frequency data.
    \item Extend NHPP/Hawkes with external regressors (credit spreads, VIX term structure).
    \item Multivariate Hawkes for cross-asset shock contagion.
    \item Deploy real-time monitoring dashboard with Hawkes intensity tracking.
\end{itemize}
\end{frame}

% ============================================================
\section{Conclusion}
% ============================================================

\begin{frame}{Conclusion}
\begin{itemize}
    \item VIX exhibits \textbf{persistent, asymmetric volatility clustering} well captured by EGARCH.
    \item Large spikes arrive at $\sim$13/year and exhibit \textbf{self-excitation} (Hawkes branching ratio 0.23).
    \item \textbf{Compound Poisson} quantifies aggregate shock risk: VaR 95\% $= 4.24$/year.
    \item \textbf{CPP out-of-sample:} $\sim$18\% forecast error; VaR bounds respected; well-calibrated predictions.
    \item \textbf{COVID regime} showed 76\% higher expected annual impact than baseline.
    \item Out-of-sample, \textbf{EWMA remains a tough benchmark} to beat for density forecasting.
    \item The reproducible pipeline (\texttt{runall.py}) with \textbf{28 unit tests} enables transparent research.
\end{itemize}

\vspace{1em}
\centering
\Large \textbf{Thank you!}\\[0.5em]
\normalsize Questions?
\end{frame}

% ============================================================
\section*{Appendix}
% ============================================================

\begin{frame}[fragile]{Appendix: Project Structure}
\small
\begin{verbatim}
Shock-Persistence-and-Shock-Frequency-in-VIX/
+-- runall.py                  # Main pipeline
+-- src/
|   +-- config.py              # Parameters
|   +-- data_pipeline.py       # Data loading
|   +-- volatility_models.py   # GARCH/EGARCH/GJR/HAR-RV
|   +-- shock_modeling.py      # HPP/NHPP/Hawkes/CPP
|   +-- forecast_evaluation.py # OOS evaluation
|   +-- visualization.py       # Plotting
+-- tests/
|   +-- test_models.py         # 28 unit tests
+-- figures/                   # Generated plots
\end{verbatim}
\end{frame}

\begin{frame}{Appendix: Key Equations}
\textbf{Log-Score:}
\[
S_t = \log f(r_t \mid \mu_t, \sigma_t^2)
\]

\textbf{Hawkes Intensity:}
\[
\lambda(t) = \mu + \sum_{t_i < t} \alpha \cdot e^{-\beta(t - t_i)}
\]

\textbf{Compound Poisson Process:}
\[
S(T) = \sum_{i=1}^{N(T)} J_i, \quad N(T) \sim \mathrm{Poisson}(\lambda T), \quad J_i \sim F
\]

\textbf{Expected Annual Impact:}
\[
\mathbb{E}[S] = \lambda \cdot \mathbb{E}[J] \cdot T
\]
\end{frame}

\begin{frame}{Appendix: References}
\small
\begin{itemize}
    \item Bollerslev, T. (1986). Generalized autoregressive conditional heteroskedasticity. \textit{J.\ Econometrics}.
    \item Nelson, D. B. (1991). Conditional heteroskedasticity in asset returns. \textit{Econometrica}.
    \item Glosten, L. R., Jagannathan, R., \& Runkle, D. E. (1993). On the relation between expected value and volatility. \textit{J.\ Finance}.
    \item Corsi, F. (2009). A simple approximate long-memory model of realized volatility. \textit{J.\ Financial Econometrics}.
    \item Hawkes, A. G. (1971). Spectra of some self-exciting and mutually exciting point processes. \textit{Biometrika}.
    \item Diebold, F. X., \& Mariano, R. S. (1995). Comparing predictive accuracy. \textit{JBES}.
\end{itemize}
\end{frame}

% ============================================================
\end{document}
